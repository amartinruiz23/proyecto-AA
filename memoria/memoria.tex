\documentclass[11pt,a4paper]{article}

% Packages
\usepackage[utf8]{inputenc}
\usepackage[spanish, es-tabla]{babel}
\usepackage{caption}
\usepackage{listings}
\usepackage{adjustbox}
\usepackage{enumitem}
\usepackage{boldline}
\usepackage{amssymb, amsmath}
\usepackage{amsthm}
\usepackage[margin=1in]{geometry}
\usepackage{xcolor}
\usepackage{soul}
\usepackage{hyperref}

% Meta
\title{Práctica 3}
\author{Pedro Bonilla Nadal}
\date{\today}

% Custom
\providecommand{\abs}[1]{\lvert#1\rvert}
\setlength\parindent{0pt}
\definecolor{Light}{gray}{.90}
\newcommand\ddfrac[2]{\frac{\displaystyle #1}{\displaystyle #2}}
% Primera derivada parcial: \pder[f]{x}
\newcommand{\pder}[2][]{\frac{\partial#1}{\partial#2}}

\begin{document}
\begin{titlepage}
\begin{center}

\vspace*{.06\textheight}
{\scshape\LARGE Universidad De Granada\par}\vspace{1.5cm} % University name
\textsc{\Large Apredizaje Automático}\\[0.5cm] % Thesis type

\rule{\textwidth}{0.4mm} \\[0.4cm] % Horizontal line
{\huge \bfseries Proyecto Final\par}\vspace{0.4cm} % Thesis title

\rule{\textwidth}{0.4mm} \\[11.5cm] % Horizontal line
 {\Large Pedro Bonilla Nadal\\Antonio Martín Ruiz}\\[1cm]
 
 {\today}

\vfill
\end{center}
\end{titlepage}

\setcounter{tocdepth}{1}
\tableofcontents
\newpage
\section{Introducción}	


\newpage
\begin{thebibliography}{9}
\bibitem{maldim}
Artículo donde explica el concepto de maldición de la dimensión: \url{https://en.wikipedia.org/wiki/Curse_of_dimensionality}

\end{thebibliography}




\end{document}
