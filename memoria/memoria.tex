\documentclass[11pt,a4paper]{article}

% Packages
\usepackage[utf8]{inputenc}
\usepackage[spanish, es-tabla]{babel}
\usepackage{caption}
\usepackage{listings}
\usepackage{adjustbox}
\usepackage{enumitem}
\usepackage{boldline}
\usepackage{amssymb, amsmath}
\usepackage{amsthm}
\usepackage[margin=1in]{geometry}
\usepackage{xcolor}
\usepackage{soul}
\usepackage{hyperref}

% Meta
\title{Práctica 3}
\author{Pedro Bonilla Nadal}
\date{\today}

% Custom
\providecommand{\abs}[1]{\lvert#1\rvert}
\setlength\parindent{0pt}
\definecolor{Light}{gray}{.90}
\newcommand\ddfrac[2]{\frac{\displaystyle #1}{\displaystyle #2}}
% Primera derivada parcial: \pder[f]{x}
\newcommand{\pder}[2][]{\frac{\partial#1}{\partial#2}}

\begin{document}
\begin{titlepage}
\begin{center}

\vspace*{.06\textheight}
{\scshape\LARGE Universidad De Granada\par}\vspace{1.5cm} % University name
\textsc{\Large Apredizaje Automático}\\[0.5cm] % Thesis type

\rule{\textwidth}{0.4mm} \\[0.4cm] % Horizontal line
{\huge \bfseries Proyecto Final\par}\vspace{0.4cm} % Thesis title

\rule{\textwidth}{0.4mm} \\[11.5cm] % Horizontal line
 {\Large Pedro Bonilla Nadal\\Antonio Martín Ruiz}\\[1cm]
 
 {\today}

\vfill
\end{center}
\end{titlepage}

\setcounter{tocdepth}{2}
\tableofcontents
\newpage



\section{Compensión del problema a resolver }

Este dataset contiene datos de la oficina del censo\cite{census}, relativos al censos de 1995. Un total de 48842 personas censadas han sido clasificadas para este censo.  De estas personas tenemos una seríe de variables con información de carácter socio económico. En particular:
\begin{itemize}
\item tenemos 6 variables de tipo numérico y entero, con valores en rangos distintos.
\item 8 variables de tipo categórico.
\item Una variable de clase que toma como valores $<50$K y $>50$k.
\item Es un problema de clasificación, ya que no tenemos información para hacer una regresión sobre la ganancia. 
\end{itemize}

El código utilizado para la resolución de la práctica se encuentra en el archivo \texttt{main.py}.  Del mismo modo, los datos limpiados se encuentran en el archivo \texttt{adults.data} y su información relativa procesada en el archivo \texttt{adult.names}.


\section{ Disvisión de los conjuntos}
Codificación de los datos de entrada para hacerlos útiles a los algoritmos
\section{Clases de funciones?}

\section{Preprocesado} 
\subsection{Valoración de las variables de interés}
\subsection{ Normalización de las variables}
\section{ Función de pérdida usada}
\section{ Técnica de Ajuste para el Modelo Lineal}
Selección de las técnica (parámetrica) y valoración de la idoneidad de la misma frente a otras alternativas
\section{Hiperparámetros y selección del modelo}
 Aplicación de la técnica especificando claramente que algoritmos se usan en la estimación de los parámetros, los hiperparámetros y el error de generalización

\section{ Error de Generalización}
\section{ Argumentar sobre la idoneidad de la función regularización usada }
\section{ Conclusiónes }
Valoración de los resultados y justificación 
( gráficas, métricas de error, análisis de residuos, etc )


que se ha obtenido la mejor de las posibles soluciones con la técnica elegida y la muestra dada. Argumentar en términos de los errores de ajuste y generalización,


\newpage
\begin{thebibliography}{9}
\bibitem{census}
Página oficial de la Oficina del censo: \url{http://www.census.gov/ftp/pub/DES/www/welcome.html}

\end{thebibliography}




\end{document}
